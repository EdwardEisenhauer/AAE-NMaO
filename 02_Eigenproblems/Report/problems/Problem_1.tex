\subsection{Problem 1}%
\label{sec:problem_1}
Compute the eigenpairs for the following matrices:
\begin{equation*}
    \matr{A}_1 = 
    \begin{bmatrix}
        1 & 0 & 0 \\
        2 & 1 & 0 \\
        0 & 0 & 3 
    \end{bmatrix},\ 
    \matr{A}_2 = 
    \begin{bmatrix}
        0 & -2 &  1 \\
        1 &  3 & -1 \\
        0 & 0 & 1 
    \end{bmatrix},\ 
    \matr{A}_3 = 
    \begin{bmatrix}
        4 & 1 & 0 \\
        1 & 4 & 1 \\
        0 & 1 & 4 
    \end{bmatrix},\ 
    \matr{A}_4 = 
    \begin{bmatrix}
         1 &  2 &  3 &  4 \\
         5 &  6 &  7 &  8 \\
         9 & 10 & 11 & 12 \\
        13 & 14 & 15 & 16
    \end{bmatrix}
\end{equation*}
Verify that the trace equals the sum of the eigenvalues, and the determinant equals
their product. Which matrix (if any) is singular?
%%%%%%%%%%%%%%%%%%%%%%%%%%%%%%%%%%%%%%%%%%%%%%%%%%%%%%%%%%%%%%%%%%%%%%%%%%%%%%%
\subsubsection*{Mathematics}
%%%%%%%%%%%%%%%%%%%%%%%%%%%%%%%%%%%%%%%%%%%%%%%%%%%%%%%%%%%%%%%%%%%%%%%%%%%%%%%
As stated in~\cite{Zarowski,Zdunek}, an \textit{eigenpair} $(\lambda, \matr{x})$ is a solution to
an \textit{eigenproblem}:
\begin{equation*}
    \matr{A}\matr{x} = \lambda\matr{x}
\end{equation*}
where the vector $\matr{x}\in\matr{C}^n$, such that $\matr{x}\neq0$ is an \textit{eigenvector}, which if transformed by the matrix
$\matr{A}\in\matr{C}^{n\times{}n}$ behaves like it would be stretched by the scalar value $\lambda\in\matr{C}$, i.e.~the
result of its transformation stays on its \textit{span}.
The above equation may be transformed into:
\begin{equation*}
    \left(\matr{A} - \lambda\matr{I}\right)\matr{x} = 0
\end{equation*}
Since we are looking for the non-zero solutions, the eigenvalues may be found by solving:
\begin{equation}
    \label{eq:eigen_value}
    \det{\left(\matr{A} - \lambda\matr{I}\right)} = 0
\end{equation}
For each of the eigenvalues $\lambda_n$, a corresponding eigenvector $\matr{x}_n$ may be then found
by solving the homogeneous linear system:
\begin{equation}
    \label{eq:eigen_vector}
    \left(\matr{A} - \lambda_n\matr{I}\right)\matr{x}_n = 0
\end{equation}

\todo[inline]{Describe matrix' trace, determinant and its eigenvalues relationship.}
\todo[inline]{Explain why the matrix singularity is important.}
%%%%%%%%%%%%%%%%%%%%%%%%%%%%%%%%%%%%%%%%%%%%%%%%%%%%%%%%%%%%%%%%%%%%%%%%%%%%%%%
\subsubsection*{Solution}
%%%%%%%%%%%%%%%%%%%%%%%%%%%%%%%%%%%%%%%%%%%%%%%%%%%%%%%%%%%%%%%%%%%%%%%%%%%%%%%
The eigenvalues for each matrix are found by finding a solution to the~\eqref{eq:eigen_value}, as follows:
\begin{equation*}
    \det{\left(\matr{A}_1 - \lambda\matr{I}\right)} = 
    \det{\left(\begin{bmatrix}
        1 & 0 & 0 \\
        2 & 1 & 0 \\
        0 & 0 & 3
    \end{bmatrix} - 
    \begin{bmatrix}
        \lambda & 0 & 0 \\
        0 & \lambda & 0 \\
        0 & 0 & \lambda 
    \end{bmatrix}\right)} =
    \det{\left(\begin{bmatrix}
        1 - \lambda & 0 & 0 \\
        2 & 1 - \lambda & 0 \\
        0 & 0 & 3 - \lambda
    \end{bmatrix}\right)}
\end{equation*}
Since there is only one non-zero value outside the diagonal:
\begin{equation*}
    (1 - \lambda)(1 - \lambda)(3 - \lambda) = 0 \rightarrow
    \lambda = \{1, 3\}
\end{equation*}
of which $\lambda=1$ is an eigenvalue of multiplicity 2.
To find the eigenvectors corresponding to each of the eigenvalues we solve~\eqref{eq:eigen_vector} with the Gauss-Jordan Elimination:
\begin{equation*}
    \left(\matr{A}_1 - \lambda_1\matr{I}\right)\matr{x}_1 = 0 \rightarrow
    \begin{amatrix}{3}{1}
        0 & 0 & 0 & 0\\
        2 & 0 & 0 & 0\\
        0 & 0 & 2 & 0
    \end{amatrix} \rightarrow
    \begin{amatrix}{3}{1}
        1 & 0 & 0 & 0\\
        0 & 0 & 1 & 0\\
        0 & 0 & 0 & 0
    \end{amatrix} \rightarrow
    \matr{x}_1 = \begin{bmatrix}
        0 \\
        x_2 \\
        0
    \end{bmatrix} \xrightarrow{x_2 = 1}
    \matr{x}_1 = \begin{bmatrix}
        0 \\
        1 \\
        0
    \end{bmatrix}
\end{equation*}
\begin{equation*}
    \left(\matr{A}_1 - \lambda_2\matr{I}\right)\matr{x}_2 = 0 \rightarrow
    \begin{amatrix}{3}{1}
        -2 & 0 & 0 & 0\\
        2 & -2 & 0 & 0\\
        0 & 0 & 0 & 0
    \end{amatrix} \xrightarrow[\substack{\frac{-R_1}{2}\\\frac{-R_2}{2}}]{R_2 + R_1}
    \begin{amatrix}{3}{1}
        1 & 0 & 0 & 0\\
        0 & 1 & 0 & 0\\
        0 & 0 & 0 & 0
    \end{amatrix} \rightarrow
    \matr{x}_2 = \begin{bmatrix}
        0 \\
        0 \\
        x_3
    \end{bmatrix} \xrightarrow{x_3 = 1}
    \matr{x}_2 = \begin{bmatrix}
        0 \\
        0 \\
        1
    \end{bmatrix}
\end{equation*}
Each of the two eigenvectors above has one arbitrary element.

Continuing with the $\matr{A}_2$:
\begin{equation*}
    \det{\left(\matr{A}_2 - \lambda\matr{I}\right)} = 
    \det{\left(\begin{bmatrix}
        0 - \lambda & -2 &  1 \\
        1 &  3 - \lambda & -1 \\
        0 & 0 & 1 - \lambda
    \end{bmatrix}\right)} \xrightarrow{R_2 + \frac{R_1}{\lambda}}
    \det{\left(\begin{bmatrix}
        - \lambda & -2 &  1 \\
        0 &  3 - \lambda - \frac{2}{\lambda} & -1 + \frac{1}{\lambda} \\
        0 & 0 & 1 - \lambda
    \end{bmatrix}\right)}
\end{equation*}
\begin{align*}
    -\lambda(3 - \lambda - \frac{2}{\lambda})(1 - \lambda) &= 0\\
    (\lambda^2 - 3\lambda + 2)(1 - \lambda) &= 0\\
    (2 - \lambda)(1 - \lambda)(1 - \lambda) &= 0\rightarrow\lambda = \{1, 2\}
\end{align*}
of which once again $\lambda=1$ is an eigenvalue of multiplicity 2. Solving~\eqref{eq:eigen_vector} to get the eigenvectors gives us:
\begin{equation*}
\begin{split}
    \left(\matr{A}_2 - \lambda_1\matr{I}\right)\matr{x}_1 = 0 \rightarrow
    \begin{amatrix}{3}{1}
        -1 & -2 & 1 & 0\\
        1 & 2 & -1 & 0\\
        0 & 0 & 0 & 0
    \end{amatrix} \xrightarrow[-R_1]{R_2 + R_1}
    \begin{amatrix}{3}{1}
        1 & 2 & -1 & 0\\
        0 & 0 & 0 & 0\\
        0 & 0 & 0 & 0
    \end{amatrix} \rightarrow\\
    \rightarrow \matr{x}_1 = \begin{bmatrix}
        -2x_2+x_3 \\
        x_2 \\
        x_3
    \end{bmatrix} \xrightarrow[x_3 = 0]{x_2 = 1}
    \matr{x}_1 = \begin{bmatrix}
        -2 \\
        1 \\
        0
    \end{bmatrix} \\
    \xrightarrow[x_3 = 1]{x_2 = 0}
    \matr{x}_1 = \begin{bmatrix}
        1 \\
        0 \\
        1
    \end{bmatrix} 
\end{split}
\end{equation*}
\begin{equation*}
\begin{split}
    \left(\matr{A}_2 - \lambda_2\matr{I}\right)\matr{x}_2 = 0 \rightarrow
    \begin{amatrix}{3}{1}
        -2 & -2 & 1 & 0\\
        1 & 1 & -1 & 0\\
        0 & 0 & -1 & 0
    \end{amatrix} \xrightarrow[\frac{-R_1}{2}]{R_2 + \frac{R_1}{2}}
    \begin{amatrix}{3}{1}
        1 & 1 & -\frac{1}{2} & 0\\
        0 & 0 & -\frac{1}{2} & 0\\
        0 & 0 & -1 & 0
    \end{amatrix} \xrightarrow[-2R_2]{\substack{R_1 - R_2\\R_3 - 2R_2}}
    \begin{amatrix}{3}{1}
        1 & 1 & 0 & 0\\
        0 & 0 & 1 & 0\\
        0 & 0 & 0 & 0
    \end{amatrix} \rightarrow\\
    \rightarrow \matr{x}_2 = \begin{bmatrix}
        -x_2 \\
        x_2 \\
        0
    \end{bmatrix} \xrightarrow{x_2 = 1}
    \matr{x}_2 = \begin{bmatrix}
        -1 \\
        1 \\
        0
    \end{bmatrix}
\end{split}
\end{equation*}
Please note, that the above eigenvectors are not normalized.

Continuing with the $\matr{A}_3$:
\begin{equation*}
    \det{\left(\matr{A}_3 - \lambda\matr{I}\right)} = 
    \det{\left(\begin{bmatrix}
        4 - \lambda & 1 &  0 \\
        1 &  4 - \lambda & 1 \\
        0 & 1 & 4 - \lambda
    \end{bmatrix}\right)} \xrightarrow{R_2 + \frac{R_1}{\lambda}}
    \det{\left(\begin{bmatrix}
        - \lambda & -2 &  1 \\
        0 &  3 - \lambda - \frac{2}{\lambda} & -1 + \frac{1}{\lambda} \\
        0 & 0 & 1 - \lambda
    \end{bmatrix}\right)}
\end{equation*}
\begin{align*}
    -\lambda(3 - \lambda - \frac{2}{\lambda})(1 - \lambda) &= 0\\
    (\lambda^2 - 3\lambda + 2)(1 - \lambda) &= 0\\
    (2 - \lambda)(1 - \lambda)(1 - \lambda) &= 0\rightarrow\lambda = \{1, 2\}
\end{align*}


Additionaly, for each matrix we calculate its trace and determinant:
\begin{equation*}
    \det{\begin{bmatrix}
        1 & 0 & 0 \\
        2 & 1 & 0 \\
        0 & 0 & 3 
    \end{bmatrix}} = 1 \cdot 1 \cdot 3 = 3
\end{equation*}
\begin{equation*}
    tr{\begin{bmatrix}
        1 & 0 & 0 \\
        2 & 1 & 0 \\
        0 & 0 & 3 
    \end{bmatrix}} = 1 + 1 + 3 = 5
\end{equation*}
