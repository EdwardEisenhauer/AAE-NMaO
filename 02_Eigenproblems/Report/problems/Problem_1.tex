\subsection{Problem 1}%
\label{sec:problem_1}

Compute the eigenpairs for the following matrices:
\begin{equation*}
    \matr{A_1} = 
    \begin{bmatrix}
        1 & 0 & 0 \\
        2 & 1 & 0 \\
        0 & 0 & 3 
    \end{bmatrix}
\end{equation*}
Verify that the trace equals to the sum of the eigenvalues, and the determinant equals
their product. Which matrix (if any) is singular?

\subsubsection*{Mathematics}

\todo{Describe what is eigenpair and what is its algebraic and geometric interpretaion.}
\todo{Describe how to compute the eigenpairs.}
\todo{Describe matrix' trace, determinant and its eigenvalues reliationship.}
\todo{Explain why the matrix singularity is important.}

\subsubsection*{Solution}

The eigenvalues for each matrix are found by finding a solution to the
$\det{\matr{A} - \lambda\matr{I}} = 0$, as follows:
\begin{equation*}
    \det{\left(\begin{bmatrix}
        1 & 0 & 0 \\
        2 & 1 & 0 \\
        0 & 0 & 3 
    \end{bmatrix} - 
    \begin{bmatrix}
        \lambda & 0 & 0 \\
        0 & \lambda & 0 \\
        0 & 0 & \lambda 
    \end{bmatrix}\right)} = 0 \rightarrow
    \det{\left(\begin{bmatrix}
        1 - \lambda & 0 & 0 \\
        2 & 1 - \lambda & 0 \\
        0 & 0 & 3 - \lambda
    \end{bmatrix}\right)} = 0
\end{equation*}
since there is only one non-zero value outside the diagonal:
\begin{equation*}
    (1 - \lambda)(1 - \lambda)(3 - \lambda) = 0 \rightarrow
    \lambda = \{1, 3\}
\end{equation*}
Additionaly, for each matrix we calculate its trace and determinant:
\begin{equation*}
    \det{\begin{bmatrix}
        1 & 0 & 0 \\
        2 & 1 & 0 \\
        0 & 0 & 3 
    \end{bmatrix}} = 1 \cdot 1 \cdot 3 = 3
\end{equation*}
\begin{equation*}
    tr{\begin{bmatrix}
        1 & 0 & 0 \\
        2 & 1 & 0 \\
        0 & 0 & 3 
    \end{bmatrix}} = 1 + 1 + 3 = 5
\end{equation*}
