\subsection{Problem 4}%
\label{sec:problem_4}
Find $\matr{A}^{100}$ by diagonalizing $\matr{A} = 
\begin{bmatrix}
    4 & 3\\
    1 & 2
\end{bmatrix}$.
\subsubsection*{Mathematics}
As stated in~\cite{Zdunek}, the matrix $\matr{A}^{n\times{}n}$ is diagonalizable if it
has $n$ distinct eigenvalues.
These can be easily found by solving~\eqref{eq:eigen_value}, as follows:
\begin{equation*}
    \det{\left(\begin{bmatrix}
        4 & 3 \\
        1 & 2 
    \end{bmatrix} - 
    \begin{bmatrix}
        \lambda & 0 \\
        0 & \lambda
    \end{bmatrix}\right)} = 0 \rightarrow
    \det{\left(\begin{bmatrix}
        4 - \lambda & 3 \\
        1 & 2 - \lambda
    \end{bmatrix}\right)} = 0
\end{equation*}
\begin{equation*}
    (4 - \lambda)(2 - \lambda) - 3 = 0 \rightarrow
    \lambda^2 - 6\lambda + 5 = 0
\end{equation*}
\begin{equation*}
    \lambda_1 = 1, \lambda_2 = 5 
\end{equation*}
Now we can define the diagonal matrix of eigenvalues $\matr{\Lambda}$:
\begin{equation*}
    \matr{\Lambda} = 
    \begin{bmatrix}
        1 & 0 \\
        0 & 5
    \end{bmatrix} 
\end{equation*} 
From these we can calculate the eigenvectors:
\begin{itemize}
    \item For $\lambda_1 = 1$:
        \begin{equation*}
            \begin{bmatrix}
                4 - 1 & 3 \\
                1 & 2 - 1
            \end{bmatrix}
            \begin{bmatrix}
                x_1 \\
                x_2
            \end{bmatrix} = 0
        \end{equation*}
        \begin{equation*}
            \begin{bmatrix}
                3 & 3 \\
                1 & 1
            \end{bmatrix}
            \begin{bmatrix}
                x_1 \\
                x_2
            \end{bmatrix} = 0
        \end{equation*}
        By solving this system we end with the eigenvector:
        \begin{equation*}
            \matr{x}_1 = 
            \begin{bmatrix}
                -1 \\
                1
            \end{bmatrix} 
        \end{equation*} 
    \item For $\lambda_2 = 5$:
    \begin{equation*}
        \begin{bmatrix}
            4 - 5 & 3 \\
            1 & 2 - 5
        \end{bmatrix}
        \begin{bmatrix}
            x_1 \\
            x_2
        \end{bmatrix} = 0
    \end{equation*}
    \begin{equation*}
        \begin{bmatrix}
            -1 & 3 \\
            1 & -3
        \end{bmatrix}
        \begin{bmatrix}
            x_1 \\
            x_2
        \end{bmatrix} = 0
    \end{equation*}
    Solving the equation for $\matr{x}_2$ eigenvector result is as such:
    \begin{equation*}
        \matr{x}_2 = 
        \begin{bmatrix}
            3 \\
            1
        \end{bmatrix} 
    \end{equation*} 
\end{itemize}
After determining the eigenpairs, the next step is to define the matrix $\matr{X}$.
\begin{equation*}
    \matr{X} = 
    \begin{bmatrix}
        -1 & 3 \\
         1 & 1
    \end{bmatrix} 
\end{equation*} 

Giving us possibility to use following equation:
\begin{equation*}
    \matr{A}^{100} = \matr{X}\matr{\Lambda}^{100}\matr{X}^{-1}
\end{equation*}
which calculated result is in solution section.
% \todo{Comapare built-in MATLAB functions and matrix diagonalizatins}

\subsubsection*{Solution}
\lstinputlisting[style=Matlab-editor]{problems/Codes/Problem_4/Dir.m}
\lstinputlisting[style=Matlab-editor]{problems/Results/Problem_4.m}
