\subsection{Problem 4}%
\label{sec:problem_4}

Find $\matr{A}^{100}$ by diagonalizing $\matr{A} = 
\begin{bmatrix}
    4 & 3\\
    1 & 2
\end{bmatrix}$.

\subsubsection*{Mathematics}

\todo{Describe when the matrix is diagonizable?}
As presented in~\cite{Zdunek}, matrix $\matr{A^{n\times n}}$ is diagonalizable if it has 
$n$ distinct eigenvalues.
These can be easily found by solving $\det{\matr{A} - \lambda\matr{I}} = 0$, as follows:
\begin{equation*}
    \det{\left(\begin{bmatrix}
        4 & 3 \\
        1 & 2 
    \end{bmatrix} - 
    \begin{bmatrix}
        \lambda & 0 \\
        0 & \lambda
    \end{bmatrix}\right)} = 0 \rightarrow
    \det{\left(\begin{bmatrix}
        4 - \lambda & 3 \\
        1 & 2 - \lambda
    \end{bmatrix}\right)} = 0
\end{equation*}
\begin{equation*}
    (4 - \lambda)(2 - \lambda) - 3 = 0 \rightarrow
    \lambda^2 - 6\lambda + 5 = 0
\end{equation*}
\begin{equation*}
    \lambda_1 = 1, \lambda_2 = 5 
\end{equation*}
With that we're able to define our diagonal matrix of eigenvalues denoted in sources as either $\Lambda$ or $D$:
\begin{equation*}
    x_2 = 
    \begin{bmatrix}
        1 & 0 \\
        0 & 5
    \end{bmatrix} 
\end{equation*} 
Next step would be defining the eigenvectors
\begin{itemize}
    \item For $\lambda_1 = 1$:
        \begin{equation*}
            \begin{bmatrix}
                4 - 1 & 3 \\
                1 & 2 - 1
            \end{bmatrix}
            \begin{bmatrix}
                x_1 \\
                x_2
            \end{bmatrix} = 0
        \end{equation*}
        \begin{equation*}
            \begin{bmatrix}
                3 & 3 \\
                1 & 1
            \end{bmatrix}
            \begin{bmatrix}
                x_1 \\
                x_2
            \end{bmatrix} = 0
        \end{equation*}
        By solving this system we end with eigenvector:
        \begin{equation*}
            x_1 = 
            \begin{bmatrix}
                -1 \\
                1
            \end{bmatrix} 
        \end{equation*} 
    \item For $\lambda_2 = 5$:
    \begin{equation*}
        \begin{bmatrix}
            4 - 5 & 3 \\
            1 & 2 - 5
        \end{bmatrix}
        \begin{bmatrix}
            x_1 \\
            x_2
        \end{bmatrix} = 0
    \end{equation*}
    \begin{equation*}
        \begin{bmatrix}
            -1 & 3 \\
            1 & -3
        \end{bmatrix}
        \begin{bmatrix}
            x_1 \\
            x_2
        \end{bmatrix} = 0
    \end{equation*}
    Solving equation for $x_2$ eigenvector result is as such:
    \begin{equation*}
        x_2 = 
        \begin{bmatrix}
            3 \\
            1
        \end{bmatrix} 
    \end{equation*} 
\end{itemize}

After determining the values of eigenpairs, next step in order would be defining matrix $P$ or $X$.
\begin{equation*}
    X = 
    \begin{bmatrix}
        -1 & 3 \\
         1 & 1
    \end{bmatrix} 
\end{equation*} 


\todo{Comapare built-in MATLAB functions and matrix diagonalizatins}

\subsubsection*{Solution}
