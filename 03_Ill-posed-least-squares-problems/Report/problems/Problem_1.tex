\subsection{Problem 1}%
\label{sec:problem_1}
Find the solution that best approximates the system of inconsistent linear equations:
\begin{tasks}(3)
  \task\label{problem:1_a} $\systeme{3x_1-x_2=4,x_1+2x_2=0,2x_1+x_2=1}$
  \task\label{problem:1_b} $\systeme{3x_1+x_2+x_3=6,
  2x_1+3x_2-x_3=1,
  2x_1-x_2+x_3=0,
  3x_1-3x_2+3x_3=8}$
  \task\label{problem:1_c} $\systeme{x_1+x_2-x_3=5,
  2x_1-x_2+6x_3=1,
  -x_1+4x_2+x_3=0,
  3x_1+2x_2-x_3=6}$
\end{tasks}
%%%%%%%%%%%%%%%%%%%%%%%%%%%%%%%%%%%%%%%%%%%%%%%%%%%%%%%%%%%%%%%%%%%%%%%%%%%%%%%
\subsubsection*{Mathematics}
%%%%%%%%%%%%%%%%%%%%%%%%%%%%%%%%%%%%%%%%%%%%%%%%%%%%%%%%%%%%%%%%%%%%%%%%%%%%%%%
The system of inconsistent linear equations comprises linearly independent equations,
whose number is greater than the number of unknown variables.
Such a system may be expressed in the form:
\begin{equation*}
  \matr{Ax}=\matr{b}, \quad \text{where} \quad
  \matr{A}\in\mathfrak{R}^{m\times{}n}, \quad
  \matr{b}\in\mathfrak{R}^m, \quad
  \matr{x}\in\mathfrak{R}^n, \quad \text{and} \quad m\geq{}n
\end{equation*}
and has no solution. We may attempt to find the best approximate solution to such a
system by solving the minimization problem:
\begin{equation*}
  \min_{\matr{x}}{\left\lVert\matr{b}-\matr{A}\matr{x}\right\rVert}_2
\end{equation*}
This problem may be expressed in the form of the \textit{linear least-squares problem}:
\begin{equation*}
  \min_{\matr{x}}{\frac{1}{2}\left\lVert\matr{b}-\matr{A}\matr{x}\right\rVert_2^2}
\end{equation*}
Following the~\cite{Zdunek}, for our system of inconsistent linear equations, we may
define an associated system of \textit{normal equations}:
\begin{equation*}
  \matr{A}^T\matr{Ax}=\matr{A}^T\matr{b}
\end{equation*}
which is always consistent and whose least-squares solution has the form:
\begin{equation}
  \label{eq:normal_approximation}
  \matr{x}={\left(\matr{A}^T\matr{A}\right)}^{-1}\matr{A}^T\matr{b}=\matr{A}^{+}b
\end{equation}
where $\matr{A}^{+}$ is the Moore-Penrose pseudoinverse.
%%%%%%%%%%%%%%%%%%%%%%%%%%%%%%%%%%%%%%%%%%%%%%%%%%%%%%%%%%%%%%%%%%%%%%%%%%%%%%%
\subsubsection*{Solution}
%%%%%%%%%%%%%%%%%%%%%%%%%%%%%%%%%%%%%%%%%%%%%%%%%%%%%%%%%%%%%%%%%%%%%%%%%%%%%%%
The solutions to the systems of linear equations above may be easily approximated
with~\nameref{algorithm:1}, implementing~\eqref{eq:normal_approximation}.
For each system, we verified our implementation with
\lstinline[style=Matlab-editor]{x - A\b}.

For the first system, we have:
\begin{equation*}
  \matr{A} = \begin{bmatrix}
    3 & -1 \\
    1 & \phantom{-}2 \\
    2 & \phantom{-}1
  \end{bmatrix} \qquad
  \matr{b} = \begin{bmatrix}
    4 \\
    0 \\
    1
  \end{bmatrix}
\end{equation*}
\lstinputlisting[style=Matlab-editor]{problems/Problem_1_a.m}
\begin{equation*}
  \matr{x} = \begin{bmatrix}
    \phantom{-}1.0482 \\
    -0.6747
  \end{bmatrix}
\end{equation*}

For the second system, we have:
\begin{equation*}
  \matr{A} = \begin{bmatrix}
    3 & \phantom{-}1 & \phantom{-}1 \\
    2 & \phantom{-}3 & -1 \\
    2 & -1 & \phantom{-}1 \\
    3 & -3 & \phantom{-}3
  \end{bmatrix} \qquad
  \matr{b} = \begin{bmatrix}
    6 \\
    1 \\
    0 \\
    8
  \end{bmatrix}
\end{equation*}
\lstinputlisting[style=Matlab-editor]{problems/Problem_1_b.m}
\begin{equation*}
  \matr{x} = \begin{bmatrix}
    -1.6667 \\
    \phantom{-}3.8333 \\
    \phantom{-}7.9167
  \end{bmatrix}
\end{equation*}

For the third system, we have:
\begin{equation*}
  \matr{A} = \begin{bmatrix}
    \phantom{-}1 & \phantom{-}1 & -1 \\
    \phantom{-}2 & -1 & \phantom{-}6 \\
    -1 & \phantom{-}4 & \phantom{-}1 \\
    \phantom{-}3 & \phantom{-}2 & -1
  \end{bmatrix} \qquad
  \matr{b} = \begin{bmatrix}
    5 \\
    1 \\
    0 \\
    6
  \end{bmatrix}
\end{equation*}
\lstinputlisting[style=Matlab-editor]{problems/Problem_1_c.m}
\begin{equation*}
  \matr{x} = \begin{bmatrix}
    \phantom{-}1.8072 \\
    \phantom{-}0.5585 \\
    -0.3810
  \end{bmatrix}
\end{equation*}
