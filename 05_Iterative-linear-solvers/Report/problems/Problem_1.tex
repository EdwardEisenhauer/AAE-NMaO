\subsection{Problem 1}%
\label{sec:problem_1}
Solve the following system with the selected iterative solvers
\begin{equation*}
  \systeme{2u-v=0,-u+2v-w=0,-v+2w-z=0,-w+2z=5}
\end{equation*}
Estimate the computational costs and convergence rates.
Start the iterations from zero-value initial guess.
%%%%%%%%%%%%%%%%%%%%%%%%%%%%%%%%%%%%%%%%%%%%%%%%%%%%%%%%%%%%%%%%%%%%%%%%%%%%%%%
\subsubsection*{Mathematics}
%%%%%%%%%%%%%%%%%%%%%%%%%%%%%%%%%%%%%%%%%%%%%%%%%%%%%%%%%%%%%%%%%%%%%%%%%%%%%%%
The principle of the iterative linear solvers of the linear system of equations
expressed as $\matr{A}\matr{x}=\matr{b}$ is to start with some initial guess
$\matr{x}^{(0)}$ and approximate the solution with each step.


\todo[inline]{Develop the iterative methods' mathematical description.}
%%%%%%%%%%%%%%%%%%%%%%%%%%%%%%%%%%%%%%%%%%%%%%%%%%%%%%%%%%%%%%%%%%%%%%%%%%%%%%%
%% Explainations

The method decomposes the matrix $\matr{A}$ into its lower triangular part $\matr{L}$, its diagonal $\matr{D}$, and its upper triangular part $\matr{U}$, 
such that $\matr{A} = \matr{L} + \matr{D} + \matr{U}$. The iteration process then uses the current and previous iterates to form the new solution vector.

The iterative step of the Gauss-Seidel method can be expressed as:
\begin{equation*}
    x^{(k+1)}_i = \frac{1}{a_{ii}}\left(b_i - \sum_{j=1}^{i-1} a_{ij} x^{(k+1)}_j - \sum_{j=i+1}^n a_{ij} x^{(k)}_j \right), \quad i = 1, 2, \ldots, n,
\end{equation*}
where:
\begin{itemize}
    \item $x^{(k)}_j$ denotes the $j$-th component of the solution vector at the $k$-th iteration.
    \item $a_{ij}$ is the element of $A$ in the $i$-th row and $j$-th column.
    \item $b_i$ is the $i$-th component of the vector $b$.
    \item The iteration for each $x^{(k+1)}_i$ uses the most recently updated values of $x^{(k+1)}$.
\end{itemize}

The Gauss-Seidel method updates each entry of the vector $x$ in sequence, directly substituting the newly computed values into subsequent equations within the same iteration. This typically allows the Gauss-Seidel method to converge faster than the Jacobi method, which updates all values based on the previous iteration's data only.

Convergence is guaranteed if the matrix $\matr{A}$ is diagonally dominant or positive definite. 
The convergence criterion is often set as the relative change in the solution vector between iterations falling below a specified tolerance.

\subsubsection*{Kaczmarz method}
Kaczmarz algorithm is differently than previously mentioned algorithms in a sense 
that it operates purely by sequentially projecting solution onto the solution spaces,
defined by each row of the matrix $\matr{A}$ in sequence. 
This approach is inherently geometric and relies on adjusting the solution to minimize 
the error perpendicular to each hyperplane at each step, whereas methods such as Gauss-Seidel decompose matrix $\matr{A}$ into its lower triangular component, 
where it iteratively refines solution by solving portion of system equation at each step using previous step value.\\
Another part differentiating the Kaczmarz method from other methods is that it is not really dependent on spectral radius although $\matr{G}$ can be generalized as $\matr{A^-1}$, but rather on properties of matrix $\matr{A}$, 
such as orthogonality, norm. Method is convergent to any consistent system where $\matr{A}$ is full rank.


Each row $i$ of the matrix $\matr{A}$ represents a hyperplane in $n$-dimensional space, and the goal of the Kaczmarz method is to find the point of intersection of these hyperplanes, which corresponds to the solution of the system.

The iterative step of the Kaczmarz method is given by:
\begin{equation}
    x_{k+1} = x_{k} + \alpha \frac{b_{i} - \langle a_{i}, x_{k} \rangle}{\|a_{i}\|_{2}^{2}} a_{i},
\end{equation}
where:
\begin{itemize}
    \item $x_{k}$ is the current estimate of the solution.
    \item $x_{k+1}$ is the next estimate of the solution.
    \item $a_{i}$ is the \(i\)-th row of $\matr{A}$ treated as a column vector.
    \item $b_{i}$ is the \(i\)-th component of the vector \(b\).
    \item $\langle \cdot, \cdot \rangle$ denotes the dot product.
    \item $\|a_{i}\|_{2}$ is the \(L_2\) norm of the vector \(a_{i}\).
    \item $i = k \mod m$, which means the rows are cycled through sequentially.
    \item $\alpha$ is the relaxation parameter.
\end{itemize}



The relaxation parameter $\alpha$ allows control over the convergence of the method. 
\begin{itemize}
    \item $\alpha = 1$ default state, each step has to project directly onto the hyperplane.
    \item $\alpha > 1$ more aggressive approach that goes beyond solution space, this can accelerate convergence if successive hyperplanes are distanced apart.
    \item $\alpha < 1$ used when method overshoots, this can happen due to noisy data set that method operates on. 
\end{itemize}


%%%%%%%%%%%%%%%%%%%%%%%%%%%%%%%%%%%%%%%%%%%%%%%%%%%%%%%%%%%%%%%%%%%%%%%%%%%%%%%
%% Spectral radius
One of steps that can allow us to check if iterations $\{x_k\}$ are convergent to $\matr{x} = \matr{A}^-1\matr{b}$ would be checking spectral radius.

\begin{equation*}
  \rho(\matr{S}^-1\matr{T}) = \max \lvert\lambda_i\rvert
\end{equation*}
\begin{enumerate}
  \item Gauss-Seidel\\
    
    $ \matr{A} = \matr{S} - \matr{T} $
    where $ \matr{S} $ is the lower triangular part of $ \matr{A} $ including the diagonal, and $ \matr{T} $ is the strict upper triangular part of $ \matr{A} $.

    The iterative formula for the Gauss-Seidel method is
    $ \matr{Sx}_{k+1} = \matr{Tx}_k + \matr{b}, $
    which can be rewritten as
    $ \matr{x}_{k+1} = \matr{Gx}_k + \matr{c}, $
    where $ \matr{G} = \matr{S}^{-1}\matr{T} $ and $ \matr{c} = \matr{S}^{-1}\matr{b} $.

    The convergence of the Gauss-Seidel method is determined by the spectral radius of $ \matr{G} $, denoted by $ \rho(\matr{G}) $. The method converges if $ \rho(\matr{G}) < 1 $.

    To find $ \rho(\matr{G}) $, we first decompose $ \matr{A} $ into $ \matr{D} $, $ \matr{L} $, and $ \matr{U} $, where $ \matr{D} $ is the diagonal of $ \matr{A} $, $ \matr{L} $ is the strict lower triangular part of $ \matr{A} $, and $ \matr{U} $ is the strict upper triangular part of $ \matr{A} $. Then, we compute $ \matr{G} $ as follows:
    $ \matr{G} = (\matr{D} + \matr{L})^{-1}\matr{U}. $

    The spectral radius $ \rho(\matr{G}) $ is the maximum absolute value of the eigenvalues of $ \matr{G} $, given by
    $ \rho(\matr{G}) = \max_i |\lambda_i|, $
    where $ \lambda_i $ are the eigenvalues of $ \matr{G} $.

    Given a matrix $ \matr{A} $, the spectral radius $ \rho(\matr{G}) $ for the Gauss-Seidel method was calculated to be approximately 0.6, indicating that the iterations will converge.
  
  \item Jacobi \\
    For the Jacobi method, the splitting of matrix $ \matr{A} $ is slightly different:
    $ \matr{A} = \matr{D} - \matr{R}, $
    where $ \matr{D} $ is the diagonal part of $ \matr{A} $, and $ \matr{R} $ is the remainder of $ \matr{A} $ (i.e., $ \matr{R} = \matr{L} + \matr{U} $).

    The iteration formula for the Jacobi method is:
    $ \matr{Dx}_{k+1} = \matr{Rx}_k + \matr{b}, $
    which can be rearranged to get the new estimate of $ \matr{x} $ at each iteration:
    $ \matr{x}_{k+1} = \matr{D}^{-1}(\matr{b} - \matr{Rx}_k). $

    In this case, the iteration matrix $ \matr{G} $ for the Jacobi method is defined as:
    $ \matr{G} = \matr{D}^{-1}\matr{R}. $

    The spectral radius of the iteration matrix, $ \rho(\matr{G}) = 0.8090$

  \item Landwebber
  In case of Landwebber our convergence criteria changes to form with pseudoinverse of $\matr{A}$ noted as $\matr{x} = \matr{A}^+ \matr{b}$:
  
  \begin{equation*}
    0 < \alfa < \frac{2}{\lambda_{max}(A^TA)}
  \end{equation*}
\end{enumerate}
%%%%%%%%%%%%%%%%%%%%%%%%%%%%%%%%%%%%%%%%%%%%%%%%%%%%%%%%%%%%%%%%%%%%%%%%%%%%%%%
\subsubsection*{Solving}

\begin{equation*}
  \matr{S}\matr{x}_{k+1} = Tx_k + b
  \matr{x}^{(k+1)} = \matr{G}\matr{x}^{(k)} + \matr{c}
\end{equation*}

As an iterative method may not converge, a good idea is to determine it for a given problem, calculating the error.
\todo[inline]{Describe the error calculation.}

Assuming we know the exact solution $\matr{x}^*$:
\begin{equation*}
  e^k = \matr{x}^k - \matr{x}^*
\end{equation*}

\begin{equation}
  e^k = \matr{G}^k * \matr{e}^0
\end{equation}

%%%%%%%%%%%%%%%%%%%%%%%%%%%%%%%%%%%%%%%%%%%%%%%%%%%%%%%%%%%%%%%%%%%%%%%%%%%%%%%
\subsubsection*{Solution}
%%%%%%%%%%%%%%%%%%%%%%%%%%%%%%%%%%%%%%%%%%%%%%%%%%%%%%%%%%%%%%%%%%%%%%%%%%%%%%%
Since the considered system of linear equations is simple and consistent, we may use the
\MATLAB's backslash operator in the form of \lstinline[style=Matlab-editor]{A\b}, to
determine the exact solution:
\begin{equation*}
  \matr{A} = \begin{bmatrix}
    \phantom{-}2 & -1 & \phantom{-}0 & \phantom{-}0 \\
    -1 & \phantom{-}2 & -1 & \phantom{-}0 \\
    \phantom{-}0 & -1 & \phantom{-}2 & -1 \\
    \phantom{-}0 & \phantom{-}0 & -1 & \phantom{-}2
  \end{bmatrix}, \qquad
  \matr{b} = \begin{bmatrix}
    0 \\
    0 \\
    0 \\
    5
  \end{bmatrix}, \qquad
  \matr{x}^* = \begin{bmatrix}
    1 \\
    2 \\
    3 \\
    4
  \end{bmatrix}
\end{equation*}
Before proceeding with calculating the approximate solution, we check the residual error
at the 100th iteration for each of the implemented algorithms to verify if the method
converges. The convergence criterion is:
\begin{equation*}
  \lim_{k\to\infty}{e} = 0
  \quad \text{if} \quad
  \max{\lvert\lambda_n\rvert}<1
  \quad \text{for} \quad
  \matr{G}^{k}
\end{equation*}
Let's verify it for each of our algorithms:
\lstinputlisting[style=Matlab-editor]{problems/Problem_1.m}

\todo[inline]{Calculate the solution error for each method.}
\todo[inline]{If the error converges --- apply the method.}
