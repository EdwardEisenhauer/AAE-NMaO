\subsection{Problem 1}%
\label{sec:problem_1}
Solve the following system with the selected iterative solvers
\begin{equation*}
  \systeme{2u-v=0,-u+2v-w=0,-v+2w-z=0,-w+2z=5}
\end{equation*}
Estimate the computational costs and convergence rates.
Start the iterations from zero-value initial guess.
%%%%%%%%%%%%%%%%%%%%%%%%%%%%%%%%%%%%%%%%%%%%%%%%%%%%%%%%%%%%%%%%%%%%%%%%%%%%%%%
\subsubsection*{Mathematics}
%%%%%%%%%%%%%%%%%%%%%%%%%%%%%%%%%%%%%%%%%%%%%%%%%%%%%%%%%%%%%%%%%%%%%%%%%%%%%%%
The principle of the iterative linear solvers of the linear system of equations
expressed as $\matr{A}\matr{x}=\matr{b}$ is to start with some initial guess
$\matr{x}^{(0)}$ and approximate the solution with each step.
\todo[inline]{Develop the iterative methods' mathematical description.}
\begin{equation*}
  \matr{S}\matr{}
  \matr{x}^{(k+1)} = \matr{G}\matr{x}^{(k)} + \matr{c}
\end{equation*}

As an iterative method may not converge, a good idea is to determine it for a given problem, calculating the error.
\todo[inline]{Describe the error calculation.}

Assuming we know the exact solution $\matr{x}^*$:
\begin{equation*}
  e^k = \matr{x}^k - \matr{x}^*
\end{equation*}

\begin{equation*}
  e^k = \matr{G}^k * \matr{e}^0
\end{equation*}

%%%%%%%%%%%%%%%%%%%%%%%%%%%%%%%%%%%%%%%%%%%%%%%%%%%%%%%%%%%%%%%%%%%%%%%%%%%%%%%
\subsubsection*{Solution}
%%%%%%%%%%%%%%%%%%%%%%%%%%%%%%%%%%%%%%%%%%%%%%%%%%%%%%%%%%%%%%%%%%%%%%%%%%%%%%%
Since the considered system of linear equations is simple and consistent, we may use the
\MATLAB's backslash operator \lstinline[style=Matlab-editor]{A\b}, to determine the
exact solution:
\begin{equation*}
  \matr{A} = \begin{bmatrix}
    \phantom{-}2 & -1 & \phantom{-}0 & \phantom{-}0 \\
    -1 & \phantom{-}2 & -1 & \phantom{-}0 \\
    \phantom{-}0 & -1 & \phantom{-}2 & -1 \\
    \phantom{-}0 & \phantom{-}0 & -1 & \phantom{-}2
  \end{bmatrix}, \qquad
  \matr{b} = \begin{bmatrix}
    0 \\
    0 \\
    0 \\
    5
  \end{bmatrix}, \qquad
  \matr{x}^* = \begin{bmatrix}
    1 \\
    2 \\
    3 \\
    4
  \end{bmatrix}
\end{equation*}
Before proceeding with finding the solution it is reasonable to enusre that the used
method will converge. The convergence criterion is:
\begin{equation*}
  \lim_{k\to\infty}{e} = 0
  \quad \text{if} \quad
  \max{\lvert\lambda_n\rvert}<1
  \quad \text{for} \quad
  \matr{G}^{k}
\end{equation*}


\todo[inline]{Calculate the solution error for each method.}
\todo[inline]{If the error converges --- apply the method.}
