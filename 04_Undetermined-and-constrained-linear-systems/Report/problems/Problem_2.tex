\subsection{Problem 2}%
\label{sec:problem_2}
Perform the forward projection of the exact solution
$\begin{bmatrix} 1 & 0 & 1 & 1 & 0 \end{bmatrix}^T$
onto the range space spanned by the columns in the matrix:
\begin{equation*}
    A = \begin{bmatrix}
        1 & 2 & 3 & 2 & 1 \\
        2 & 4 & 4 & 6 & 2 \\
        3 & 6 & 6 & 9 & 6 \\
        1 & 2 & 4 & 5 & 3
\end{bmatrix}
\end{equation*}
Assuming the linear forward projection model, try to estimate the true solution to
$\matr{Ax} = \matr{b}$ given $\matr{A}$ and $\matr{b}$.
Then change the $a_{21}$ entry from 2 to 0 and repeat the estimation.
Explain the difference.
Compute the residual and solution errors.
Which algorithm gives the best estimate and why?
Which metrics are best to solbe this problem?
%%%%%%%%%%%%%%%%%%%%%%%%%%%%%%%%%%%%%%%%%%%%%%%%%%%%%%%%%%%%%%%%%%%%%%%%%%%%%%%
\subsubsection*{Mathematics}
%%%%%%%%%%%%%%%%%%%%%%%%%%%%%%%%%%%%%%%%%%%%%%%%%%%%%%%%%%%%%%%%%%%%%%%%%%%%%%%
First step towards solving the problem would be obtaining matrix b by calculating.
\begin{equation*}
    \matr{A}\matr{x} = \matr{b} \rightarrow 
    \begin{bmatrix}
        1 & 2 & 3 & 2 & 1 \\
        2 & 4 & 4 & 6 & 2 \\
        3 & 6 & 6 & 9 & 6 \\
        1 & 2 & 4 & 5 & 3
    \end{bmatrix}
    \begin{bmatrix}
        1 \\
        0 \\
        1 \\
        1 \\
        0
    \end{bmatrix} =
    \begin{bmatrix}
        6 \\
        12 \\
        18 \\
        10
\end{bmatrix}
\end{equation*}
With calculated solution we're able to create estimated "true" solution.
\begin{equation*}
    \matr{A}^T\matr{A}\matr{\hat{x}} = \matr{A}^T\matr{b} \rightarrow
    \matr{\hat{x}} = (\matr{A}^T\matr{A})^-1 \matr{A}^T\matr{b} = \matr{A}^+\matr{b}
\end{equation*}
Since our matrix is not square we are forced to use pseudo-ivnerse which is denoted as $A^+$.\\
This way least squares solution has form:
\begin{equation*}
    \matr{\hat{x}} = \matr{A}^+ \matr{b}
\end{equation*}
Calculating $\matr{A^+}$ requires using SVD of base matrix $\matr{\matr{A}}$
\begin{equation*}
    \matr{A^+} = \matr{V} 
    \begin{bmatrix}
        \matr{\Sigma^{-1}} && 0 \\
        0 && 0
    \end{bmatrix}
    \matr{U}^H = \sum_{i=1}^{r} \sigma_i^{-1} \matr{v}_i\matr{u}_i^H \in \matr{\Im}^{N \times M} 
\end{equation*}
Pseudo-inverse of matrix $\matr{A}$ was calculated using MATLAB

\begin{equation*}
    \matr{A^+ = }
    \begin{bmatrix}
        0.0273  &  0.0545  &  0.0909  & -0.2273 \\
        0.0545  &  0.1091  &  0.1818  & -0.4545 \\
        0.0364  &  0.0727  & -0.2121  &  0.3636 \\
        0.0636  &  0.1273  & -0.1212  &  0.1364 \\
        -0.2000 &  -0.4000 &   0.3333 &  -1.32 * 10^-16 \\
    \end{bmatrix}
\end{equation*}

With pseudo-inverse calculated we're able to calculate "real" $\\matr{\hat{x}}$ for solution $\matr{b}$
\begin{equation*}
    \matr{\hat{x}} = \matr{A^+}\matr{b} = 
    \begin{bmatrix}
        0.0273  &  0.0545  &  0.0909  & -0.2273 \\
        0.0545  &  0.1091  &  0.1818  & -0.4545 \\
        0.0364  &  0.0727  & -0.2121  &  0.3636 \\
        0.0636  &  0.1273  & -0.1212  &  0.1364 \\
        -0.2000 &  -0.4000 &   0.3333 &  -1.32 * 10^{-16} \\
    \end{bmatrix}
    \begin{bmatrix}
        6 \\
        12 \\
        18 \\
        10
    \end{bmatrix} =
    \begin{bmatrix}
        0.1818 \\ 
        0.3636 \\
        0.9091 \\
        1.0909 \\
        0.0000
    \end{bmatrix}
\end{equation*}

Next step would be to modify matrix $\matr{A}$ in such way that element $a_{21} = 0$.  
\begin{equation*}
    \matr{A} = 
    \begin{bmatrix}
        1 & 2 & 3 & 2 & 1 \\
        0 & 4 & 4 & 6 & 2 \\
        3 & 6 & 6 & 9 & 6 \\
        1 & 2 & 4 & 5 & 3
    \end{bmatrix}
\end{equation*}
Recalculating matrix $\matr{b}$ results in:
\begin{equation*}
    \matr{A}\matr{x} = \matr{b} = 
    \begin{bmatrix}
        6 \\
        10\\
        18\\
        10
    \end{bmatrix}
\end{equation*}

With newly formed $\matr{A}$ pseudo inverse of this matrix is:
\begin{equation*}
    \matr{A^+} = 
    \begin{bmatrix}
        1.0000  & -0.5000 & 0 & 0 \\
        -0.1111 &   0.2222 & 0.2222 & -0.5556 \\
        0.2778  & -0.0556 & -0.2222 & 0.3889 \\
        0.2222  &  0.0556 & -0.1111 & 0.1111 \\
        -1.0000 &  -6.11 * 10^{-16} & 0.3333 & -8.32 * 10^{-17}
    \end{bmatrix}
\end{equation*}

This time real solution is close to base solution
\begin{equation*}
    \matr{\hat{x}} = \matr{A^+} \matr{b} =
    \begin{bmatrix}
        1.0000 \\
        -8.88 * 10^{-16} \\
        1.0000 \\
        1.0000 \\
        -2.61 * 10^{-15}
    \end{bmatrix}
\end{equation*}

Discrepancy as to why true solution differs when we change $a_{21}$ entry to 0 from 2 is quite trivial, if the step of calculating the rank of matrix wasn't skipped.\\
$\left(\begin{gmatrix}
    1 & 2 & 3 & 2 & 1 \\
    2 & 4 & 4 & 6 & 2 \\
    3 & 6 & 6 & 9 & 6 \\
    1 & 2 & 4 & 5 & 3\end{gmatrix}\right.
    \begin{gmatrix}[q]
    6 \\
    12 \\
    18 \\
    10
    \rowops
    \add[\cdot\left(-2\right)]{0}{1}
    \add[\cdot\left(-3\right)]{0}{2}
    \add[\cdot\left(-1\right)]{0}{3}
    
    \end{gmatrix}$
    
    $\left(\begin{gmatrix}
    1 & 2 & 3 & 2 & 1 \\
    0 & 0 & -2 & 2 & 0 \\
    0 & 0 & -3 & 3 & 3 \\
    0 & 0 & 1 & 3 & 2\end{gmatrix}\right.
    \begin{gmatrix}[q]
    6 \\
    0 \\
    0 \\
    4
    \rowops
    \mult{1}{\cdot \left(\frac{-1}{2}\right)}
    
    \end{gmatrix}$
$\left(\begin{gmatrix}
    1 & 2 & 3 & 2 & 1 \\
    0 & 0 & 1 & -1 & 0 \\
    0 & 0 & -3 & 3 & 3 \\
    0 & 0 & 1 & 3 & 2\end{gmatrix}\right.
    \begin{gmatrix}[q]
    6 \\
    0 \\
    0 \\
    4
    \rowops
    \add[\cdot\left(-3\right)]{1}{0}
    \add[\cdot3]{1}{2}
    \add[\cdot\left(-1\right)]{1}{3}
    
    \end{gmatrix}$
    
    $\left(\begin{gmatrix}
    1 & 2 & 0 & 5 & 1 \\
    0 & 0 & 1 & -1 & 0 \\
    0 & 0 & 0 & 0 & 3 \\
    0 & 0 & 0 & 4 & 2\end{gmatrix}\right.
    \begin{gmatrix}[q]
    6 \\
    0 \\
    0 \\
    4
    \rowops
    \swap{2}{3}
    
    \end{gmatrix}$
    
    $\left(\begin{gmatrix}
    1 & 2 & 0 & 5 & 1 \\
    0 & 0 & 1 & -1 & 0 \\
    0 & 0 & 0 & 4 & 2 \\
    0 & 0 & 0 & 0 & 3\end{gmatrix}\right.
    \begin{gmatrix}[q]
    6 \\
    0 \\
    4 \\
    0
    \rowops
    \mult{2}{\cdot \frac{1}{4}}
    
    \end{gmatrix}$
    
    $\left(\begin{gmatrix}
    1 & 2 & 0 & 5 & 1 \\
    0 & 0 & 1 & -1 & 0 \\
    0 & 0 & 0 & 1 & \frac{1}{2} \\
    0 & 0 & 0 & 0 & 3\end{gmatrix}\right.
    \begin{gmatrix}[q]
    6 \\
    0 \\
    1 \\
    0
    \rowops
    \add[\cdot\left(-5\right)]{2}{0}
    \add{2}{1}
    
    \end{gmatrix}$
    
    $\left(\begin{gmatrix}
    1 & 2 & 0 & 0 & \frac{-3}{2} \\
    0 & 0 & 1 & 0 & \frac{1}{2} \\
    0 & 0 & 0 & 1 & \frac{1}{2} \\
    0 & 0 & 0 & 0 & 3\end{gmatrix}\right.
    \begin{gmatrix}[q]
    1 \\
    1 \\
    1 \\
    0
    \end{gmatrix}$
    \\
For base matrix $\matr{A}$ rank is 3, meaning it is not a full rank matrix, hence solution using least squares is not accurate.\\

$\left(\begin{gmatrix}
    1 & 2 & 3 & 2 & 1 \\
    0 & 4 & 4 & 6 & 2 \\
    3 & 6 & 6 & 9 & 6 \\
    1 & 2 & 4 & 5 & 3\end{gmatrix}\right.
    \begin{gmatrix}[q]
    6 \\
    10 \\
    18 \\
    10
    \rowops
    \add[\cdot\left(-3\right)]{0}{2}
    \add[\cdot\left(-1\right)]{0}{3}
    
    \end{gmatrix}$
    
    $\left(\begin{gmatrix}
    1 & 2 & 3 & 2 & 1 \\
    0 & 4 & 4 & 6 & 2 \\
    0 & 0 & -3 & 3 & 3 \\
    0 & 0 & 1 & 3 & 2\end{gmatrix}\right.
    \begin{gmatrix}[q]
    6 \\
    10 \\
    0 \\
    4
    \rowops
    \mult{1}{\cdot \frac{1}{4}}
    
    \end{gmatrix}$
    
    $\left(\begin{gmatrix}
    1 & 2 & 3 & 2 & 1 \\
    0 & 1 & 1 & \frac{3}{2} & \frac{1}{2} \\
    0 & 0 & -3 & 3 & 3 \\
    0 & 0 & 1 & 3 & 2\end{gmatrix}\right.
    \begin{gmatrix}[q]
    6 \\
    \frac{5}{2} \\
    0 \\
    4
    \rowops
    \add[\cdot\left(-2\right)]{1}{0}
    
    \end{gmatrix}$
    
    $\left(\begin{gmatrix}
    1 & 0 & 1 & -1 & 0 \\
    0 & 1 & 1 & \frac{3}{2} & \frac{1}{2} \\
    0 & 0 & -3 & 3 & 3 \\
    0 & 0 & 1 & 3 & 2\end{gmatrix}\right.
    \begin{gmatrix}[q]
    1 \\
    \frac{5}{2} \\
    0 \\
    4
    \rowops
    \mult{2}{\cdot \left(\frac{-1}{3}\right)}
    
    \end{gmatrix}$
    
    $\left(\begin{gmatrix}
    1 & 0 & 1 & -1 & 0 \\
    0 & 1 & 1 & \frac{3}{2} & \frac{1}{2} \\
    0 & 0 & 1 & -1 & -1 \\
    0 & 0 & 1 & 3 & 2\end{gmatrix}\right.
    \begin{gmatrix}[q]
    1 \\
    \frac{5}{2} \\
    0 \\
    4
    \rowops
    \add[\cdot\left(-1\right)]{2}{0}
    \add[\cdot\left(-1\right)]{2}{1}
    \add[\cdot\left(-1\right)]{2}{3}
    
    \end{gmatrix}$
    
    $\left(\begin{gmatrix}
    1 & 0 & 0 & 0 & 1 \\
    0 & 1 & 0 & \frac{5}{2} & \frac{3}{2} \\
    0 & 0 & 1 & -1 & -1 \\
    0 & 0 & 0 & 4 & 3\end{gmatrix}\right.
    \begin{gmatrix}[q]
    1 \\
    \frac{5}{2} \\
    0 \\
    4
    \rowops
    \mult{3}{\cdot \frac{1}{4}}
    
    \end{gmatrix}$
    
    $\left(\begin{gmatrix}
    1 & 0 & 0 & 0 & 1 \\
    0 & 1 & 0 & \frac{5}{2} & \frac{3}{2} \\
    0 & 0 & 1 & -1 & -1 \\
    0 & 0 & 0 & 1 & \frac{3}{4}\end{gmatrix}\right.
    \begin{gmatrix}[q]
    1 \\
    \frac{5}{2} \\
    0 \\
    1
    \rowops
    \add[\cdot\left(\frac{-5}{2}\right)]{3}{1}
    \add{3}{2}
    
    \end{gmatrix}$
    
    $\left(\begin{gmatrix}
    1 & 0 & 0 & 0 & 1 \\
    0 & 1 & 0 & 0 & \frac{-3}{8} \\
    0 & 0 & 1 & 0 & \frac{-1}{4} \\
    0 & 0 & 0 & 1 & \frac{3}{4}\end{gmatrix}\right.
    \begin{gmatrix}[q]
    1 \\
    0 \\
    1 \\
    1
    \end{gmatrix}$
    \\
Modified $\matr{A}$ rank is 4 being full rank matrix. This implies unique solution for least squares problem given that $\matr{b}$ is in the column space of $\matr{A}$.
%%%%%%%%%%%%%%%%%%%%%%%%%%%%%%%%%%%%%%%%%%%%%%%%%%%%%%%%%%%%%%%%%%%%%%%%%%%%%%%
\subsubsection*{Solution}
Using MATLAB script we performed calculations using pseudo inverse, FOCUSS and M-FOCUSS Methods
\lstinputlisting[style=Matlab-editor]{problems/Problem_2.m}
%%%%%%%%%%%%%%%%%%%%%%%%%%%%%%%%%%%%%%%%%%%%%%%%%%%%%%%%%%%%%%%%%%%%%%%%%%%%%%%

