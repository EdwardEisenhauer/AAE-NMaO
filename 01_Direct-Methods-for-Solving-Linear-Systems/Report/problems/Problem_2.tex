\subsection{Problem~2}%
\label{problem:2}

Solve the following system of linear equations using the Gaussian Elimination with
pivoting:
\begin{equation*}
    \systeme{x_1+x_2+x_3=1,x_1+x_2+2x_3=2,x_1+2x_2+2x_3=1}
\end{equation*}

Explain why the Gaussian elimination without pivoting does not work.

\subsubsection*{Mathematics}

As stated at the end of the mathematical introduction to the \nameref{problem:1},
the \nameref{algorithm:gaussian_elimination} requires all the diagonal entries of the coefficient
matrix to be non-zero, to avoid the division by zero during the row normalization.
While in manual computations such a division would be noticed, in \MATLAB their result
is \texttt{Inf} and the algorithm proceeds with replacing subsequent coefficients with
this value, resulting in gibberish output.

This behaviour can be avoided by using a \textit{pivoting}.
The basic idea is to find the row with the maximum absolute value in the zeroed column below the
pivot and swap it with the one currently considered.
This is called a \textit{partial pivoting}.
Searching in the whole submatrix spanned between pivot and $\matr{A_{n,m}}$ ---
 interchanging not only rows but also columns --- is called a \textit{complete pivoting}.

\subsubsection*{Solution}

As shown below, upon zeroing the elements of the first column below the first pivot $a_{1,1}$, the second pivot $a_{2,2}$ is zeroed as well.
To proceed with the Gaussian elimination, one must swap the second and third rows, performing a partial pivoting.

\begin{equation*}
\begin{split}
    \begin{amatrix}{1}{1}
        \matr{A} & \matr{b}
    \end{amatrix} &=
    \begin{amatrix}{3}{1}
        1 & 1 & 1 & 1\\
        1 & 1 & 2 & 2\\
        1 & 2 & 2 & 1
    \end{amatrix} \xrightarrow[\substack{R_2-R_1\\R_3-R_1}]{}
    \begin{amatrix}{3}{1}
        1 & 1 & 1 & 1\\
        0 & 0 & 1 & 1\\
        0 & 1 & 1 & 0
    \end{amatrix} \xrightarrow[R_2\leftrightarrow R_3]{}
    \begin{amatrix}{3}{1}
        1 & 1 & 1 & 1\\
        0 & 1 & 1 & 0\\
        0 & 0 & 1 & 1
    \end{amatrix}\rightarrow\\
    &\rightarrow
    \systeme{x_1+x_2+x_3=1,x_2+x_3=0,x_3=1}
    \rightarrow
    \systeme{x_1+x_2+x_3=\phantom{-}1,x_2=-1,x_3=\phantom{-}1}
    \rightarrow
    \begin{cases}
    x_1=\phantom{-}1\\
    x_2=-1\\
    x_3=\phantom{-}1
    \end{cases}
\end{split}
\end{equation*}

Our implementation of \nameref{algorithm:gaussian_elimination_with_partial_pivoting} in \MATLAB\footnote{Algorithm \hyperref[algorithm:gaussian_elimination_with_partial_pivoting]{\texttt{gaussian\_elimination\_with\_partial\_pivoting}}} yields the same result:
\lstinputlisting[style=Matlab-editor]{problems/Problem_2.m}
