\subsection{Problem 1}

Solve the system and find the pivots when:
\begin{equation*}
    \systeme{2u-v=0,-u+2v-w=0,-v+2w-z=0,-w+2z=5}
\end{equation*}

\subsubsection*{Mathematics}
The solution to a linear system like the one presented above may be obtained via Gauss-Jordan Elimination. An augmented
matrix
$\matr{B} = \begin{amatrix}{1}{1}
    \matr{A} & \matr{b}
\end{amatrix}$, where $\matr{A}$ is a coefficient matrix, and $\matr{b}$ is a
column vector of constant terms, is assembled. Then, the algorithm implemented
in \ref{algorithm:5} is performed. Note, that none of the $a_{ii}$ can be equal
to zero for this algorithm to work properly\footnote{Explanation of this
property is presented in \ref{problem:2}}.

\subsubsection*{Solution}

Manual calculations were performed. In the following transformations a row
normalization is presented above the arrow and then a zeroing is below it.

\begin{equation*}
\begin{split}
    \begin{amatrix}{1}{1}
        \matr{A} & \matr{b}
    \end{amatrix} = 
    \begin{amatrix}{4}{1}
        \phantom{-}2 & -1 & \phantom{-}0 & \phantom{-}0 & 0\\
        -1 & \phantom{-}2 & -1 & \phantom{-}0 & 0\\
        \phantom{-}0 & -1 & \phantom{-}2 & -1 & 0\\
        \phantom{-}0 & \phantom{-}0 & -1 & \phantom{-}2 & 5
    \end{amatrix} \xrightarrow[R_2 + R_1]{\frac{1}{2}R_1} 
    \begin{amatrix}{4}{1}
        \phantom{-}1 & -\frac{1}{2} & \phantom{-}0 & \phantom{-}0 & 0\\
        \phantom{-}0 & \phantom{-}\frac{3}{2} & -1 & \phantom{-}0 & 0\\
        \phantom{-}0 & -1 & \phantom{-}2 & -1 & 0\\
        \phantom{-}0 & \phantom{-}0 & -1 & \phantom{-}2 & 5
    \end{amatrix} \xrightarrow[\substack{R_1+\frac{1}{2}R_2\\R_3+R_2}]{\frac{2}{3}R_2} \\
    \xrightarrow[R_3+\frac{2}{3}R_2]{\frac{2}{3}R_2}
    \begin{amatrix}{4}{1}
        \phantom{-}1 & \phantom{-}0 & -\frac{1}{3} & \phantom{-}0 & 0\\
        \phantom{-}0 & \phantom{-}1 & -\frac{2}{3} & \phantom{-}0 & 0\\
        \phantom{-}0 & \phantom{-}0 & \phantom{-}\frac{4}{3} & -1 & 0\\
        \phantom{-}0 & \phantom{-}0 & -1 & \phantom{-}2 & 5
    \end{amatrix} \xrightarrow[\substack{R_1+\frac{1}{3}R_3\\R_2+\frac{2}{3}R_3\\R_4+R_3}]{\frac{3}{4}R_3}
    \begin{amatrix}{4}{1}
        \phantom{-}1 & \phantom{-}0 & \phantom{-}0 & -\frac{1}{4} & 0\\
        \phantom{-}0 & \phantom{-}1 & \phantom{-}0 & -\frac{1}{2} & 0\\
        \phantom{-}0 & \phantom{-}0 & \phantom{-}1 & -\frac{3}{4} & 0\\
        \phantom{-}0 & \phantom{-}0 & \phantom{-}0 & -\frac{4}{3} & 5
    \end{amatrix} \xrightarrow[\substack{R_1+\frac{1}{4}R_4\\R_2+\frac{1}{2}R_4\\R_3+\frac{3}{4}R_4}]{-\frac{3}{4}R_4}\\
    \xrightarrow[\substack{R_1+\frac{1}{4}R_4\\R_2+\frac{1}{2}R_4\\R_3+\frac{3}{4}R_4}]{-\frac{3}{4}R_4}
    \begin{amatrix}{4}{1}
        \phantom{-}1 & \phantom{-}0 & \phantom{-}0 & \phantom{-}0 & 1\\
        \phantom{-}0 & \phantom{-}1 & \phantom{-}0 & \phantom{-}0 & 2\\
        \phantom{-}0 & \phantom{-}0 & \phantom{-}1 & \phantom{-}0 & 3\\
        \phantom{-}0 & \phantom{-}0 & \phantom{-}0 & \phantom{-}1 & 4
    \end{amatrix} \rightarrow
    \begin{cases}
    u=1\\
    v=2\\
    w=3\\
    z=4
    \end{cases}
\end{split}
\end{equation*}

Above result was verified with the Algorithm~\ref{algorithm:5}:
\lstinputlisting[style=Matlab-editor]{problems/Problem_1.m}
