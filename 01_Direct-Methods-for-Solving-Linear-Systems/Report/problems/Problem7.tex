\subsection{Problem 7}

Let $\mathbf{A} = \left[a_{ij}\right] \in\mathbb{R}^{N\times N}$ with, $a_{ij} = \frac{1}{i + j - 1}$ (Hilbert  matrix). For  $N=5$, perform  the LU factorization of the matrix $\mathbf{A}$. Then, compute det($\mathbf{A}$).
\subsubsection*{Mathematics}
Hilbert matrices are an example of ill-conditioned matrices, which -- with their high $\kappa$ -- makes the numerical computations highly unstable. For our  $\mathbf{H} = \left[h_{ij}\right] \in\mathbb{R}^{5\times 5}$:

\begin{equation*}
    \matr{H} = 
    \begin{bmatrix}
               1 & \frac{1}{2} & \frac{1}{3} & \frac{1}{4} & \frac{1}{5} \\
     \frac{1}{2} & \frac{1}{3} & \frac{1}{4} & \frac{1}{5} & \frac{1}{6} \\
     \frac{1}{3} & \frac{1}{4} & \frac{1}{5} & \frac{1}{6} & \frac{1}{7} \\
     \frac{1}{4} & \frac{1}{5} & \frac{1}{6} & \frac{1}{7} & \frac{1}{8} \\
     \frac{1}{5} & \frac{1}{6} & \frac{1}{7} & \frac{1}{8} & \frac{1}{9} 
    \end{bmatrix}
\end{equation*}

$\kappa(\matr{H})\approx4.7\cdot10^5$ which is a huge value similar to this in the Problem 4.
\subsubsection*{Solution}
\lstinputlisting[style=Matlab-editor]{problems/Problem7.m}
The above results may look promising, but a determinant calculated both by \MATLAB and our algorithms is of the order of $4\cdot10^{-12}$, which for numerical computations is highly not useful.